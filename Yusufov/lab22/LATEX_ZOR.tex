\documentclass[a4paper, 10pt]{book}
\usepackage[english, russian]{babel}
\usepackage[utf8]{inputenc}
\usepackage{amssymb}
\usepackage{upgreek}
\usepackage{amsmath}
\usepackage{amsfonts}
\usepackage[left=10mm, top=15mm,
            right=20mm, bottom=15mm,
            nohead, nofoot{geometry}

\setlength{\headheight}{0mm}
\setlength{\headsep}{0mm}
\setcounter{page}{349}

\begin{document}
    \begin{center}
        $\S4.$ \scriptsize НЕКОТОРЫЕ ПРИЛОЖЕНИЯ ИНТЕГРАЛА
    \end{center}
    \par\small\textbf{2. Длина пути.} Пусть частица движется в пространстве $\mathbb{R}^{3}$, и пусть известен закон ее движения \textit{r(t) = (x(t), y(t), z(t))}, где \textit{x(t), y(t), z(t)} — прямоугольные декартовы координаты точки в момент времени \textit{t}
    \par Мы хотим определить длину \textit{l}[\textit{a, b}] пути, пройденного точкой за промежуток времени \textit{a} \leqslant \textit{t} \leqslant \textit{b}.
    \par Уточним некоторые понятия.
    \par\textsc{Определение 1.} \textit{Путём} в пространстве $\mathbb{R}^{3}$ называется отображение \textit{t} $\mapsto$ \textit{(x(t), y(t), z(t))} числового промежутка в пространство $\mathbb{R}^{3}$, задаваемое непрерывными на этом промежутке функциями \textit{x(t), y(t), z(t)}.
    \par\textsc{Определение 2.} Если \textit{t} $\mapsto$ \textit{(x(t), y(t), z(t))} есть путь, для которого областью изменения параметра \textit{t} является отрезок [\textit{a}, \textit{b}], то точки
    \[\textit{A = (x(a), y(a), z(a)), B = (x(b), y(b), z(b))}\]
    пространства $\mathbb{R}^{3}$ называются соответственно \textit{началом} и \textit{концом} пути.
    \par\textsc{Определение 3.} Путь называется \textit{замкнутым}, если он имеет и начало, и конец и эти точки совпадают.
    \par\textsc{Определение 4.} Если Г: \textit{I} $\rightarrow$ $\mathbb{R}^{3}$ — путь, то образ Г\textit{(I)} промежутка \textit{I} в пространстве $\mathbb{R}^{3}$ называется \textit{носителем} пути.
    \parНоситель абстрактного пути может оказаться вовсе не тем, что мы хотели бы назвать линией. Существуют примеры путей, носители которых, например, содержат целый трехмерный куб (так называемые «кривые» Пеано).Однако если функции \textit{x(t), y(t), z(t)} достаточно регулярны (как, например,в случае механического движения, когда они дифференцируемы), то ничего противоречащего нашей интуиции, как можно строго проверить, заведомо не произойдет.
    \par\textsc{Определение 5.} Путь Г: \textit{I} $\rightarrow$ $\mathbb{R}^{3}$, для которого отображение \textit{I} $\rightarrow$ Г(\textit{I}) взаимно однозначно, называется простым путем или параметризованной кривой, а его носитель — кривой в $\mathbb{R}^{3}$.
    \par\textsc{Определение 6.} Замкнутый путь Г: [\textit{a, b}] $\rightarrow$ $\mathbb{R}^{3}$ называется простым замкнутым путем или простой замкнутой кривой, если путь Г: [\textit{a, b}] $\rightarrow$ $\mathbb{R}^{3}$ является простым.
    \parЗначит, простой путь отличается от произвольного пути тем, что придвижении по его носителю мы не возвращаемся в прежние точки, т. е. непересекаем свою траекторию нигде, кроме, быть может, ее конца, если простой путь замкнут.
    \par\textsc{Определение 7.} Путь Г: \textit{I} $\rightarrow$ $\mathbb{R}^{3}$ называется путем \textit{данного класса гладкости}, если задающие его функции \textit{x(t), y(t), z(t)} принадлежат указанному классу.
    \par(Например, классу \textit{C}[\textit{a, b}], ${C}^{(1)}$[\textit{a, b}] или ${C}^{(k)}$[\textit{a, b}].)
    \par\textsc{Определение 8.} Путь Г: \textit{I} $\rightarrow$ $\mathbb{R}^{3}$ называется \textit{кусочно гладким}, еслиотрезок [\textit{a, b}] можно разбить на конечное число отрезков, на каждом из которых соответствующее ограничение отображения Г задается непрерывно дифференцируемыми функциями.

    \newpage
    \begin{center}
        \scriptsize ГЛ. VI. ИНТЕГРАЛ
    \end{center}
    \par Именно гладкие пути, т. е. пути класса ${C}^{(1)}$, и кусочно гладкие пути мы и будем сейчас рассматривать.
    \par Вернемся к исходной задаче, которую теперь можно сформулировать как задачу определения длины гладкого пути Путь Г: \textit{I} $\rightarrow$ $\mathbb{R}^{3}$.
    \par Наши исходные представления о длине \textit{l}[$\upalpha$, $\upbeta$] пути, пройденного в промежуток времени $\upalpha$ \leqslant \textit{t} \leqslant $\upbeta$, таковы: во-первых, если $\upalpha$ \textless $\upbeta$ \textless $\upgamma$, то 
    \[\textit{l}[\upalpha,\upgamma] = \textit{l}[\upalpha,\upbeta] + \textit{l}[\upbeta,\upgamma]\]
    и, во-вторых, если \textit{$\upupsilon$(t) = ($\dot x$(t),  ̇$\dot y$(t),  ̇$\dot z$(t))} есть скорость точки в момент \textit{t}, то
    \[\inf\limits_{x\in [\upalpha,\upbeta]}|\textit{$\upupsilon$(t)}|(\upbeta - \upalpha) \leqslant \textit{l}[\upalpha,\upbeta] \leqslant \sup\limits_{x\in [\upalpha,\upbeta]} |\textit{$\upupsilon$(t)}|(\upbeta - \upalpha)\]
    \par\ Таким образом, если функции \textit{x(t), y(t), z(t)} непрерывно дифференцируемы на [\textit{a, b}], то в силу утверждения 1 мы однозначно приходим к формуле
    \[\textit{l}[\textit{a, b}] \int_a^b\limits{|\textit{$\upupsilon$(t)}|}\textit{dt} = \int_a^b\limits{}\sqrt{\dot x^2(t) + \dot y^2(t) + \dot z^2(t)} \textit{dt},\eqno (3)\]
    \par которую и принимаем теперь как определение длины гладкого пути Г: [\textit{a, b}] $\rightarrow$ $\mathbb{R}^{3}$.
    \par Если \textit{z(t)} $\equiv$ 0, то носитель пути лежит в плоскости и формула (3) приобретает вид
    \[\textit{l}[\textit{a, b}] = \int_a^b\limits{}\sqrt{\dot x^2(t) + \dot y^2(t)} \textit{dt},\eqno (4)\]
    \par \textsc{Пример 3}. Опробуем формулу (4) на знакомом объекте. Пусть точка движется в плоскости по закону
    \[x = R\cos\2\pi t, \ \ x = R\sin\2\pi t. \eqno (5)\]
    \par За промежуток времени [0, 1] точка один раз пробежит окружность радиуса \textit{R}, т. е. пройдет путь длины 2$\pi$\textit{R}, если длина окружности вычисляется по этой формуле.
    Проведем расчет по формуле (4):
    \[\textit{l} [0, 1] \int_0^1\limits{}\sqrt{(-2\pi R \sin2\pi t)^2 + (2 \pi R \cos2\pi t)^2} dt = 2 \pi R.\]
    \par Несмотря на ободряющее совпадение результатов, проведенное рассуждение содержит некоторые логические пробелы, на которые стоит обратить внимание.
    \par Функции $\cos$ $\upalpha$ и $\sin$ $\upalpha$, если принять их школьное определение, суть декартовы координаты образа \textit{p} точки \textit{$p_0$} = (1,0) при повороте на угол $\upalpha$.
    \par Величина $\upalpha$ с точностью до знака измеряется длиной дуги окружности \textit{$x^2 + y^2 = 1$}, заключенной между \textit{$p_0$} и \textit{p}. Таким образом, при этом подходе к

    \newpage
    \begin{center}
        $\S4.$ \scriptsize НЕКОТОРЫЕ ПРИЛОЖЕНИЯ ИНТЕГРАЛА
    \end{center}
    \par тригонометрическим функциям их определение опирается на понятие длины дуги окружности и, значит, вычисляя выше длину окружности, мы совершили в известном смысле логический круг, задав параметризацию окружности в виде (5).
    \par Однако эта трудность, как мы сейчас увидим, не принципиальная, ибо параметризацию окружности можно задать, вовсе не прибегая к тригонометрическим функциям.
    \par Рассмотрим задачу о вычислении длины графика функции \textit{$y = f(x)$}, определенной на некотором отрезке [\textit{a, b}] $\subset$ $\mathbb{R}$. Имеется в виду вычисление длины пути Г: [\textit{a, b}] $\rightarrow$ $\mathbb{R}^{2}$, имеющего специальный вид параметризации
    \[\textit{x} \mapsto \textit{(x, f(x))},\]
    из которого можно заключить, что отображение Г: [\textit{a, b}] $\rightarrow$ $\mathbb{R}^{2}$ взаимно однозначно. Значит, по определению 5 график функции есть кривая в $\mathbb{R}^{2}$. Формула (4) в данном случае упрощается, поскольку, полагая в ней \textit{$x=t$}, \textit{$y=f(t)$}, получаем
    \[\textit{l} [0, 1] = \int_a^b\limits{} \sqrt{1 + [\textit{f'(x)}]^2} \textit{dx}. \eqno{(6)}\]
    \par В частности, если рассмотреть полуокружность
    \[\textit{y} = \sqrt{1 - x^2}, \ \  -1 \leqslant x \leqslant 1,\]
    \par окружности \textit{$x^2 + y^2 = 1$}, то для неё получим
    \[l = \int\limits_{-1}^{+1} \sqrt{1 + [\frac{-x}{\sqrt{1-x^2}}]^2} dx = \int\limits_{-1}^{1} \frac{dx}{\sqrt{1-x^2}} . \eqno{(7)}\]
    \par Но под знаком последнего интеграла стоит неограниченная функция и, значит, он не существует в традиционном, изученном нами смысле. Означает ли это, что полуокружность не имеет длины? Пока это только означает, что указанная параметризация полуокружности не удовлетворяет условиям непрерывности функций $\dot x$(t), $\dot y$(t), при которых была выписана формула (4), а значит, и формула (6). Поэтому нам следует либо подумать о расширении понятия интеграла, с тем чтобы интеграл в (7) получил определенный смысл, либо перейти к параметризации, удовлетворяющей условиям применимости формулы (6).
    \par Заметим, что если взятую параметризацию рассматривать на любом отрезке вида [-1 + $\updelta$, 1 - $\updelta$], где -1 $\leq$ -1 + $\updelta$ $\leq$ 1 - $\updelta$ $\leq$ 1, то на нем формула (6) применима и по ней находим длину
    \[l [-1 + \updelta, 1 - \updelta] = \int\limits_{-1 + \updelta}^{1 - \updelta} \frac{dx}{\sqrt{1 - x^2}}\]
    дуги окружности, лежащей над отрезком [-1 + $\updelta$, 1 - $\updelta$].
\end{document}