\documentclass{article}
\usepackage[T2A]{fontenc}
\usepackage{amsmath}

\newcommand\hr{\par\vspace{-.5\ht\strutbox}\noindent\hrulefill\par}

\begin{document}
\thispagestyle{empty}
НЕКОТОРЫЕ ЗАДАЧИ КОЛЛОКВИУМОВ \mbox{}\hfill 631

\hr
\bigskip
Известно, что всегда $O(f)+o(f)=O(f)$, и $o(f)+o(f)=o(f)$, и $2 o(f)=o(f)$ при фиксированной базе. Следует ли отсюда, что $o(f) \equiv 0$ ?

\textbf{6.} Известно, что произведение двух или любого конечного числа бесконечно малых является функцией бесконечно малой. Приведите пример, показывающий, что для бесконечных произведений это уже не всегда так.

\textbf{7. }Зная степенное разложение функции $e^x$, найдите методом неопределенных коэффициентов (или иначе) несколько первых членов (или все) степенного разложения функции $\ln (1+x)$.

\textbf{8. }Вычислите $\exp A$, когда $A$-одна из матриц
$$
\left(\begin{array}{ll}
0 & 0 \\
0 & 0
\end{array}\right),\left(\begin{array}{ll}
0 & 1 \\
0 & 0
\end{array}\right),\left(\begin{array}{lll}
0 & 1 & 0 \\
0 & 0 & 1 \\
0 & 0 & 0
\end{array}\right),\left(\begin{array}{lll}
1 & 0 & 0 \\
0 & 2 & 0 \\
0 & 0 & 3
\end{array}\right) .
$$

\textbf{9}. Сколько членов ряда для $e^x$ надо взять, чтобы получить многочлен, позволяющий вычислять $e^x$ на отрезке $[-1,2]$ с точностью до $10^{-3}$ ?

\textbf{10. } Зная степенные разложения функций $\sin x$ и $\cos x$, найдите методом неопределенных коэффициентов (или иначе) несколько первых членов (или все) степенного разложения функции $\operatorname{tg} x$ в окрестности точки $x=0$.

\textbf{11. }Длину стягивающего земной шар по экватору пояска увеличили на 1 метр, после чего поясок натянули, подперев вертикальным столбиком. Какова примерно высота столбика, если радиус Земли $\approx 6400$ км.?

\textbf{12.} Вычислите
$$
\lim _{x \rightarrow \infty}\left(e \cdot\left(1+\frac{1}{x}\right)^{-x}\right)^x
$$

\textbf{13. }Нарисуйте эскизы графиков следующих функций:

a) $\log _{\cos x} \sin x$
b) $\operatorname{arctg} \frac{x^3}{(1-x)(1+x)^2}$.

\bigskip
\bgroup
\obeylines
\textbf{Дифференциальное исчисление функций
одной переменной}
\egroup
\bigskip

1. Покажите, что если вектор ускорения $a(t)$ в любой момент $t$ ортогонален вектору $\boldsymbol{v}(t)$ скорости движения, то величина $|\boldsymbol{v}(t)|$ остается постоянной.
2. Пусть $(x, t)$ и $(\tilde{x}, \tilde{t})$-соответственно координата и время движущейся

\newpage
\thispagestyle{empty}

НЕКОТОРЫЕ ЗАДАЧИ КОЛЛОКВИУМОВ \mbox{}\hfill 632

\hr
\bigskip

дифференцируема на $\mathbb{R}$, то $f^{\prime}$ непрерывна в любой точке $a \in \mathbb{R}$. По теореме Лагранжа
$$
\frac{f(x)-f(a)}{x-a}=f^{\prime}(\xi)
$$
где $\xi$-точка между $a$ и $x$. Тогда если $x \rightarrow a$, то $\xi \rightarrow a$. По определению,
$$
\lim _{x \rightarrow a} \frac{f(x)-f(a)}{x-a}=f^{\prime}(a)
$$
и поскольку этот предел существует, то существует и равен ему предел правой части формулы Јагранжа, т. е. $f^{\prime}(\xi) \rightarrow f^{\prime}(a)$ при $\xi \rightarrow a$. Непрерывность $f^{\prime}$ в точке $a$ «доказана». Где ошибка?

\textbf{4.} Пусть функция $f$ имеет $n+1$ производную в точке $x_0$, и пусть $\xi=$ $=x_0+\theta_x\left(x-x_0\right)$ - средняя точка в формуле Јагранжа остаточного члена $\frac{1}{n !} f^{(n)}(\xi)\left(x-x_0\right)^n$, так что $0<\theta_x<1$. Покажите, что $\theta_x \rightarrow \frac{1}{n+1}$ при $x \rightarrow x_0$, если $f^{(n+1)}\left(x_0\right) \neq 0$.

\textbf{5.} а) Если функция $f \in C^{(n)}([a, b], \mathbb{R})$ в $n+1$ точке отрезка $[a, b]$ имеет нули, то на этом отрезке имеется по крайней мере один нуль функции $f^{(n)}$ производной $f$ порядка $n$.
b) Покажите, что полином $P_n(x)=\frac{d^n\left(x^2-1\right)^n}{d x^n}$ на отрезке $[-1,1]$ имеет $n$ корней. (Указание: $x^2-1=(x-1)(x+1)$ и $P_n^{(k)}(-1)=P_n{ }^{(k)}(1)=0$ при $k=0, \ldots, n-1$.)

\textbf{6.} Вспомните геометрический смысл производной и покажите, что если функция $f$ определена и дифференцируема на интервале $I$ и $[a, b] \subset I$, то функция $f^{\prime}$ (даже не будучи непрерывной!) принимает на отрезке $[a, b]$ все значения между $f^{\prime}(a)$ и $f^{\prime}(b)$.

\textbf{7.} Докажите неравенство
$$
a_1^{\alpha_1} \ldots a_n^{\alpha_n} \leqslant \alpha_1 a_1+\ldots+\alpha_n a_n
$$
где числа $a_1, \ldots, a_n, \alpha_1, \ldots, \alpha_n$ неотрицательны и $\alpha_1+\ldots+\alpha_n=1$.

\textbf{8.} Покажите, что
$$
\lim _{n \rightarrow \infty}\left(1+\frac{z}{n}\right)^n=e^x(\cos y+i \sin y) \quad(z=x+i y)
$$
поэтому естественно считать, что $e^{i y}=\cos y+i \sin y$ (формула Эйлера) и
$$
e^z=e^x e^{i y}=e^x(\cos y+i \sin y)
$$

\textbf{9.} Найдите форму поверхности жидкости, равномерно вращающейся в стакане.

\newpage
\thispagestyle{empty}

НЕКОТОРЫЕ ЗАДАЧИ КОЛЛОКВИУМОВ \mbox{}\hfill 633

\hr
\bigskip

\textbf{10.} Покажите, что касательная к эллипсу $\frac{x^2}{a^2}+\frac{y^2}{b^2}=1$ в точке $\left(x_0, y_0\right)$ имеет уравнение $\frac{x x_0}{a^2}+\frac{y y_0}{b^2}=1$ и что световые лучи от источника, помещенного в одном из фокусов $F_1=\left(-\sqrt{a^2-b^2}, 0\right), F_2=\left(\sqrt{a^2-b^2}, 0\right)$ эллипса с полуосями $a>b>0$, собираются эллиптическим зеркалом в другом фокусе.

\textbf{11.} Частица без предварительного разгона под действием силы тяжести начинает скатываться с вершины ледяной горки эллиптического профиля. Уравнение профиля: $x^2+5 y^2=1, y \geqslant 0$. Рассчитайте траекторию движения частицы до ее приземления.

\textbf{12.} Средним порядка $\alpha$ чисел $x_1, x_2, \ldots, x_n$ называют величину
$$
s_\alpha\left(x_1, x_2, \ldots, x_n\right)=\left(\frac{x_1^\alpha+x_2^\alpha+\ldots+x_n^\alpha}{n}\right)^{1 / \alpha}
$$
В частности, при $\alpha=1,2,-1$ получаем соответственно средяее арифметическое, среднее квадратичное и среднее гармоническое этих чисел.

Будем считать, что все числа $x_1, x_2, \ldots, x_n$ неотрицательны, а если степень $\alpha<0$, то будем предполагать, что они даже положительны.

a) Используя неравенство Гёльдера, покажите, что если $\alpha<\beta$, то
$$
s_\alpha\left(x_1, x_2, \ldots, x_n\right) \leqslant s_\beta\left(x_1, x_2, \ldots, x_n\right)
$$
причем равенство имеет место, лишь когда. $x_1=x_2=\ldots=x_n$.

b) Покажите, что при стремлении $\alpha$ к нулю величина $s_\alpha\left(x_1, x_2, \ldots, x_n\right)$ стремится к $\sqrt[n]{x_1 x_2 \ldots x_n}$, т.е. к средчему геометрическому этих чисел.

С учетом результата задачи а) отсюда, например, следует классическое неравенство между средним геометрическим и средним арифметическим неотрицательных чисел (напишите его).

c) Если $\alpha \rightarrow+\infty$, то $s_\alpha\left(x_1, x_2, \ldots, x_n\right) \rightarrow \max \left\{x_1, x_2, \ldots, x_n\right\}$, а при $\alpha \rightarrow$ $\rightarrow-\infty$ величина $s_\alpha\left(x_1, x_2, \ldots, x_n\right)$ стремится к меньшему из рассматриваемых чисел, т. е. к $\min \left\{x_1, x_2, \ldots, x_n\right\}$. Докажите это. функция времени). Считаем, что это непрерывно дифференцируемая функция на промежутке $a \leqslant t \leqslant b$.

a) Можно ли, ссылаясь на теорему Лагранжа о среднем, утверждать, что на $[a, b]$ найдется момент $\xi$, такой что $r(b)-r(a)=\boldsymbol{r}^{\prime}(\xi) \cdot(b-a)$ ? Поясните ответ примерами.

b) Пусть Convex $\left\{r^{\prime}\right\}$ - выпуклая оболочка множества (концов) векторов $\boldsymbol{r}^{\prime}(t), t \in[a, b]$. Покажите, что найдется вектор $v \in$ Convex $\left\{\boldsymbol{r}^{\prime}\right\}$, такой что $\boldsymbol{r}(b)-\boldsymbol{r}(a)=\boldsymbol{v} \cdot(b-a)$

c) Соотношение $|\boldsymbol{r}(b)-\boldsymbol{r}(a)| \leqslant \sup \left|\boldsymbol{r}^{\prime}(t)\right| \cdot|b-a|$, где верхняя грань берется по $t \in[a, b]$, имеет очевидный физический смысл. Какой? Докажите это неравенство как общий математический факт, развивающий классическую теорему Лагранжа о конечном приращении.


\end{document}