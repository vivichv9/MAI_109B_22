\documentclass[a5paper, 16pt]{book}

\usepackage[left=10mm, top=15mm, right=20mm, bottom=15mm, nohead, nofoot]{geometry}
\usepackage[english, russian]{babel}
\usepackage[utf8]{inputenc}
\usepackage{wasysym}
\usepackage{amssymb}

\setlength{\headheight}{0mm}
\setlength{\headsep}{0mm}
\setcounter{page}{264}

\date{26.02.2023}
\author{Олег Концебалов}

\begin{document}
    \begin{center}
        \begin{spacing}
            ГЛ. V. ДИФФЕРЕНЦИАЛЬНОЕ ИСЧИСЛЕНИЕ
            \noindent\rule{\textwidth}{1pt}
		\end{spacing}
    \end{center}
    \parПредположим, что утверждение доказано для порядков $n = k - 1 \geq 1$. Покажем, что тогда оно справедливо также для порядка $n = k \geq 2$.
    \parЗаметим предварительно, что поскольку 
    $$\varphi^{(k)} (x_0) = \bigg(\varphi^{(k - 1)}\bigg)^{'} (x_0) = \lim_{E \ni x\to x_0} \frac{\varphi^{(k - 1)} (x) - \varphi^{(k - 1)} (x_0)}{x - x_0} ,$$
    то существование $\varphi^{(k)} (x_0)$ предполагает, что функция $\varphi^{(k - 1)} (x)$ определена на $E$ хотя бы вблизи точки $x_0$. Уменьшая, если нужно, отрезок $E$, можно заранее считать, что функции $\varphi (x) , \varphi^{'} (x) , . . . , \varphi^{(k - 1)} (x)$, где $k \geq 2$, определены на всем отрезке $E$ с концом $x_0$. Поскольку $k \geq 2$, то функция $\varphi (x)$ имеет на $E$ производную $\varphi^{'} (x)$ и по условию
    $$(\varphi^{'})^{'} (x_0) = . . . = (\varphi^{'})^{(k - 1)} (x_0) = 0.$$
    \parТаким образом, по предположению индукции
    $$\varphi^{'} (x) = o\bigg((x - x_0)^{k - 1}\bigg) \quad \textrm{при} \quad  x\to x_0 , x\in E.$$
    Тогда, используя теорему Лагранжа, получаем
    $$\varphi (x) = \varphi (x) - \varphi(x_0) = \varphi^{'} (\xi) (x - x_0) = \alpha (\xi) (\xi - x_0)^{k - 1} (x - x_0),$$
    где $\xi$ \textbf{"---} точка, лежащая между $x_0$ и $x$, т.е. $|\xi - x_0| < |x - x_0|,$ а $\alpha (\xi) \to 0$ при $\xi \to E, \xi \in E.$ Значит, при $x \to x_0, x \in E$ одновременно будем иметь $\xi \to E, \xi \in E \textrm{ и } \alpha (\xi) \to 0,$ и поскольку
    $$|\varphi (x)| \leq |\alpha (\xi)| |x - x_0|^{k - 1} |x - x_0|,$$
    то проверено, что 
    $$\varphi (x) = o\bigg((x - x_0)^{k} \bigg) \quad \textrm{при} \quad x \to x_0, x \in E.$$
    \par Таким образом, утверждение леммы 2 проверено принципом математической индукции. $\blacktriangleright$
    \parСоотношение (33) называется $\textit{локальной формулой Тейлора}$, поскольку указанный в нем вид остаточного члена (так называемая $\textit{форма Пеано}$)
    \begin{flushright} 
    $r_n (x_0 ; x) = o ((x - x_0)^{n}) \qquad \qquad \qquad \qquad \qquad \textrm{(34)}$ 
    \end{flushright}

    \newpage
 
    \begin{flushleft}
        \begin{spacing}
            $\S3.$ \small{ОСНОВНЫЕ ТЕОРЕМЫ ДИФФЕРЕНЦИАЛЬНОГО ИСЧИСЛЕНИЯ}
            \noindent\rule{\textwidth}{1pt}
		\end{spacing}
    \end{flushleft}
    позволяет делать заключения только об асимптотическом характере связи полиному Тейлора и функции при $x \to x_0,  x \in E.$
    \parФормула (33) удобна, таким образом, при вычислении пределов и описании асимптотики функции при $x \to x_0, x \in E,$ но она не может служить для приближенного вычисления значений функции до тех пор, пока нет фактической оценки величины $r_n (x_0 ; x) = o((x - x_0)^{n}).$
    \parПодведем итоги. Мы опредилили полином Тейлора
    $$P_n (x_0 ; x) = f (x_0) + \frac{f^{'} (x_0)}{1!} (x - x_0) + . . . + \frac{f^{(n)} (x_0)}{n!} (x - x_0)^{n} ,$$
    написали формулу Тейлора
    $$f (x) = f(x_0) + \frac{f^{'} (x_0)}{1!} (x - x_0) + . . . + \frac{f^{(n)} (x_0)}{n!} (x - x_0)^{n} + r_n (x_0 ; x)$$
    и получили следующие ее важнейшие конкретизации:
    \par\textit{Если f имеет производную порядка n + 1 в интервале с концами $x_0, x,$ то}
    $$f (x) = f (x_0) +  \frac{f^{'} (x_0)}{1!} (x - x_0) + . . . + \frac{f^{(n)} (x_0)}{n!} (x - x_0)^{n} + \frac{f^{(n + 1)} (\xi)}{(n + 1)!} (x - x_0)^{n + 1} , \textrm{(35)}$$
    \textit{где $\xi$ \textbf{"---} точка, лежащая между $x_0 \textit{ и } x.$}
    \par\textit{Если f имеет в точке $x_0$ все производные до порядка $n \geq 1$ включительно, то}
    $$f (x) = f (x_0) +  \frac{f^{'} (x_0)}{1!} (x - x_0) + . . . + \frac{f^{(n)} (x_0)}{n!} (x - x_0)^{n} + o ((x - x_0)^{n}). \; \textrm{(36)}$$
    \parСоотношение (35), называемое \textit{формулой Тейлора с остаточным членом в форме Лагрнажа}, очевидно, является обобщением теоремы Лагранжа, в которую оно превращается при $n = 0.$
    \parСоотношение (36), называемое \textit{формулой Тейлора с остаточным членом в форме Пеано}, очевидно, является обобщением определения дифференцируемости функции в точке, в которое оно переходит при $n = 1.$

    \newpage

    \begin{center}
        \begin{spacing}
            ГЛ. V. ДИФФЕРЕНЦИАЛЬНОЕ ИСЧИСЛЕНИЕ
            \noindent\rule{\textwidth}{1pt}
		\end{spacing}
    \end{center}
    \parЗаметим, что формула (35) практически всегда более содержательна, ибо, с одной стороны, как мы видели, она позволяет оценивать обсолютную величину остаточного члена, а с другой, например, при ограниченности $f^{(n + 1)} (x)$ в окрестности $x_0$ из нее вытекает также асимптотическая формула
    $$f (x) = f (x_0) +  \frac{f^{'} (x_0)}{1!} (x - x_0) + . . . + \frac{f^{(n)} (x_0)}{n!} (x - x_0)^{n} + O \big((x - x_0)^{n + 1}\big). \; \textrm{(37)}$$
    Так что для бесконечно дифференцируемых функций, с которыми в подавляющем большинстве случаев имеет дело классический анализ, формула (35) содержит в себе локальную формулу (36).
    \parВ частности, на основании формулы (37) и разнообразных выше примеров 3 \textbf{--} 10 можно теперь выписать слудующую таблицу асимптотических формул при $x \to 0:$
    $$e^{x} = 1 + \frac{1}{1!} x + \frac{1}{2!} x^{2} + . . . + \frac{1}{n!} x^{n} + O(x^{n + 1}) ,$$
    $$\textrm{cos} \, x = 1 - \frac{1}{2!} x^{2} + \frac{1}{4!} x^{4} - . . . + \frac{(-1)^{n}}{(2n)!} x^{2n} + O(x^{2n + 2}),$$
    $$\textrm{sin} \, x = x - \frac{1}{3!} x^{3} + \frac{1}{5!} x^{5} - . . . + \frac{(-1)^{n}}{(2n + 1)!} x^{2n + 1} + O(x^{2n + 3}),$$
    $$\textrm{ch} \, x = 1 + \frac{1}{2!} x^{2} + \frac{1}{4!} x^{4} + . . . + \frac{1}{(2n)!} x^{2n} + O(x^{2n + 2}),$$
    $$\textrm{sh} \, x = x + \frac{1}{3!} x^{3} + \frac{1}{5!} x^{5} + . . . + \frac{1}{(2n + 1)!} x^{2n + 1} + O(x^{2n + 3}),$$
    $$\textrm{ln} (1 + x) = x - \frac{1}{2} x^{2} + \frac{1}{3} x^{3} - . . . + \frac{(-1)^{n - 1}}{n} x^{n} + O(x^{n + 1}),$$
    $$(1 + x)^{\alpha} = 1 + \frac{\alpha}{1!} x + \frac{\alpha (\alpha - 1)}{2!} x^{2} + . . . + \frac{\alpha (\alpha - 1) . . . (\alpha - n + 1)}{n!} x^{n} + O(x^{n + 1}).$$
    \parРассмотрим теперь еще некоторые примеры использования формулы Тейлора.
    \par\textbf{Пример 11.} Напишем полином, позволяющий вычислять значение функции sin\,$x$ на отрезке $-1 \leq x \leq 1$ с абсолютной погрешностью, не превышающей $10^{-3}$.

    \newpage

    \begin{flushleft}
        \begin{spacing}
            $\S3.$ \small{ОСНОВНЫЕ ТЕОРЕМЫ ДИФФЕРЕНЦИАЛЬНОГО ИСЧИСЛЕНИЯ}
            \noindent\rule{\textwidth}{1pt}
		\end{spacing}
    \end{flushleft}
    \parВ качестве такого многочлена можно взять тейлоровский многочлен подходящей степени, получаемый разложением функции sin\,$x$ в окрестности точки $x_0 = 0.$ Поскольку
    $$\textrm{sin} \, x = x - \frac{1}{3!} x^{3} + \frac{1}{5!} x^{5} - . . . + \frac{(-1)^{n}}{(2n + 1)!} x^{2n + 1} + 0 \cdot x^{2n + 2} + r_2n+2 (0 ; x) ,$$
    где по формуле Лагранжа
    $$r_2n+2 (0 ; x) = \frac{\textrm{sin} \bigg(\xi + \frac{\pi}{2} (2n + 3) \bigg)}{(2n + 3)!} x^{2n + 3},$$
    то при $|x| \leq 1$
    $$|r_2n + 2 (0 ; x)| \leq \frac{1}{(2n + 3)!}.$$
    Но $\frac{1}{(2n + 3)!} < 10^{-3}$ при $n \geq 2.$ Таким образом, с нужной точностью на отрезке $|x| \leq 1$ имеем $\textrm{sin} \, x \approx x - \frac{1}{3!} x^{3} + \frac{1}{5!} x^{5}.$
    \par\textbf{Пример 12.} Покажем, что $\textrm{tg} \, x = x + \frac{1}{3} x^{3} + o(x^{3})$ при $x \to 0.$ Имеем 
    $$\textrm{tg} ^{'} \, x = \textrm{cos} ^{-2} \, x ,$$
    $$\textrm{tg} ^{''} \, x = 2\textrm{cos} ^{-3} \, x \, \textrm{sin} \, x ,$$
    $$\textrm{tg} ^{'''} \, x = 6\textrm{cos} ^{-4} \, x \, \textrm{sin} ^{2} \, x + 2\textrm{cos} ^{-2} \, x .$$
    \parТаким образом, $\textrm{tg} \, 0 = 0,$ $\textrm{tg} ^{'} \, 0 = 1,$ $\textrm{tg} ^{''} \, 0 = 0,$ $\textrm{tg} ^{'''} \, 0 = 2$ и написанное соотношение следует из локальной формулы Тейлора.
    \par\textbf{Пример 13.} Пусть $\alpha > 0.$ Исследуем сходимость ряда $\sum\limits_{n = 1}^\infty \textrm{ln\,cos\,} \frac{1}{n^{\alpha}} .$ При $\alpha > 0 \frac{1}{n^{\alpha}} \to 0,$ когда $n \to \infty$. Оценим порядок члена ряда
    $$\textrm{ln\,cos\,} \frac{1}{n^{\alpha}} = \textrm{ln\,} \bigg(1 - \frac{1}{2!} \cdot \frac{1}{n^{2\alpha}} + o\bigg(\frac{1}{n^{2\alpha}}\bigg)\bigg) = - \frac{1}{2} \cdot \frac{1}{n^{2\alpha}} + o\bigg(\frac{1}{n^{2\alpha}}\bigg) .$$
    \parТаким образом, мы имеем знакопостоянный ряд, члены которого эквивалентны членам ряда $\sum\limits_{n = 1}^\infty \frac{-1}{2n^{2\alpha}} .$ Поскольку последний ряд сходится только при $\alpha > \frac{1}{2},$ то в указанной области $\alpha > 0$ исходный ряд сходится лишь при $\alpha > \frac{1}{2}$ (см. задачу 16b).
    
\end{document}