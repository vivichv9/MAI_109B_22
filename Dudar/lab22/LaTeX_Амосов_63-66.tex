\documentclass[a5paper, 16pt]{book}
\usepackage[left=15mm, top=15mm, right=20mm, bottom=15mm]{geometry}
\usepackage[english, russian]{babel}
\usepackage[utf8]{inputenc}
\usepackage{wasysym}
\usepackage{amssymb}
\pagestyle{empty}
\date{23.03.2023}
\author{Юрий Дударь}

\begin{document}
\section*{Лекция 7}
\subsection*{7.1. \ Классические и квантовые случайные процессы}
Целью лекции будет рассказ о том, как классические винеровский и пуассоновский случайные процессы вкладываются в симметричное (бозонное) пространство Фока. Что касается винеровского процесса, идеология такого вложения следует разложению Винера-Ито пространства $L^{2}$-функционалов от броуновского движения [12]. Вложение пуассоновского процесса было предложено в пионерской работе [32] .

Всюду ниже $T$ обозначено одно из множеств $\mathbb{R}, \mathbb{R}_{+}, \mathbb{Z}$ или $\mathbb{Z}_{+}$.\\

\noindent\textbf{Определение 7.1.} \ Однопараметрическое множество случайных величин $\left\{\xi_{t}, \ t \in T\right\}$ называется (\textit{классическим}) \textit{случайным процессом}, если задано совместное распределение вероятностей:
$$\operatorname{Pr}\left(\xi_{t_{1}} \in B_{1}, \ldots, \xi_{t_{n}} \in B_{n}\right)$$ для любого выбора индексов $t_{j} \in T$ и подмножеств $B_{j} \in \mathcal{B}(\mathbb{R})$.\\

\noindent\textbf{Определение 7.2.} \ Процесс $\left\{\xi_{t}, \ t \in T\right\}$ называется \textit{случайным процессом с независимыми приращениями}, \ если случайные величины $\xi_{t_{1}} \ -\xi_{s_{1}}$ \ и \ $\xi_{t_{2}} \ -\xi_{s_{2}}$ \ независимы при любом выборе непересекающихся интервалов $\left(s_{1}, t_{1}\right) \cap\left(s_{2}, t_{2}\right)$.\\

\noindent\textbf{Пример 7.1.} \ Случайный процесс с независимыми приращениями $\left\{\xi_{t}, \ t \in \mathbb{R}_{+}\right\} \ $называется \textit{винеровским}, если $\left.\xi_{t}-\xi_{s} \in \mathcal{N}(0, t-s)\right), s<t$, то есть распределение вероятностей $\xi_{t}-\xi_{s}$ является гауссовским:

$$\operatorname{Pr}\left(\xi_{t}-\xi_{s} \in B\right)=\frac{1}{\sqrt{2 \pi(t-s)}} \int_{B} \exp \left(-\frac{x^{2}}{2(t-s)}\right) d x$$ $s<t, B \in \mathcal{B}\left(\mathbb{R}_{+}\right)$\\

\noindent\textbf{Пример 7.2.} \ Случайный процесс с независимыми приращениями $\left\{\xi_{t}, \ t \in \mathbb{R}_{+}\right\} \ $называется \textit{пуассоновским}, если $\xi_{t}-\xi_{s} \in \mathcal{P}(t-s), s<t$, то есть распределение вероятностей $\xi_{t}-\xi_{s}$ является пуассоновским:
$$\operatorname{Pr}\left(\xi_{t}-\xi_{s}=k\right)=e^{-t+s} \frac{(t-s)^{k}}{k !},$$ $s<t, \ k=0,1,2, \ldots$ 

\newpage

Определение классических случайных процессов с независимыми приращениями может быть перенесено на квантовый случай в следующей форме:\\

\noindent\textbf{Определение 7.3.} \ Пара $\left(\left\{X_{t}, \ t \in T\right\}, \rho\right)$, состоящая из семейства наблюдаемых $X_{t} \in \mathfrak{A}$ и состояния $\rho \in \mathfrak{S}$, называется \textit{квантовым случайным процессом с независимыми приращениями}, если приращения процесса $X_{s t}=X_{t}-X_{s}$ коммутируют (наблюдаемые совместимы):
$$
\left[X_{s_{1} t_{1}}, X_{s_{2} t_{2}}\right]=0
$$ для непересекающихся интервалов $\left(s_{1}, t_{1}\right) \cap\left(s_{2}, t_{2}\right)=\emptyset$ и классические случайные величины $\xi_{s_{j} t_{j}}$ с совместным распределением, определяемым формулой (см. определение 3.15):
$$
\operatorname{Pr}\left(\xi_{s_{1} t_{1}} \in B_{1}, \ldots, \xi_{s_{n} t_{n}} \in B_{n}\right)=\operatorname{Tr}\left(\rho E_{1}\left(B_{1}\right) \ldots E_{n}\left(B_{n}\right)\right),
$$ где $\left(E_{j}\right)$ - проекторозначные меры, отвечающие наблюдаемым $\left(X_{s_{j} t_{j}}\right)$, независимы.

\subsection*{7.2. \ Симметричное пространство Фока}
Рассмотрим тензорное произведение $H^{\otimes^{n}}$, состоящее из $n$ копий гильбертова пространства $H$. \ Скалярное произведение в \ $H^{\otimes^{n}}$ \ задаётся на элементарных тензорах формулой
$$
\left\langle f_{1} \otimes \cdots \otimes f_{n}, g_{1} \otimes \cdots \otimes g_{n}\right\rangle_{H^{\otimes^{n}}}=\prod_{j=1}^{n}\left\langle f_{j}, g_{j}\right\rangle_{H},
$$
$f_{j}, g_{j} \in H$, и продолжается затем на всё пространство по линейности. 

Определим ортогональный проектор $P_{s}: H^{\otimes^{n}} \rightarrow H^{\otimes^{n}}$ по формуле
$$
P_{s} e_{1} \otimes e_{2} \otimes \cdots \otimes e_{n}=\frac{1}{n !} \sum_{s \in S} e_{s(1)} \otimes \cdots \otimes e_{s(n)},
$$
где суммирование ведётся по множеству $S$, состоящему из всех перестановок множества $\{1, \ldots, n\}$ и $e_{j} \in H$.\\

\noindent\textbf{Определение 7.4.} \ Подпространство $H^{\otimes_{s}^{n}}=P_{s} H^{\otimes^{n}}$ называется \textit{симметризованным тензорным произведением} $n$ копий пространства $H$.\\

\noindent\textbf{Определение 7.5.} \ Гильбертово пространство:
$$
F(H)=\{\mathbb{C} \Omega\} \oplus \bigoplus_{n=1}^{+\infty} H^{\otimes_{s}^{n}}
$$

\newpage

\noindent называется \textit{симметричным (бозонным) \ пространством Фока.}  Фиксированный вектор $\Omega$ называется \textit{вакуумным}, пространство $H-$ \textit{одночастным}, а пространства $H^{\otimes_{s}^{n}}$ - $n$-\textit{частичными}.\\

Ранее мы рассматривали модель квантового гармонического осциллятора. Нашей целью теперь будет построение модели бесконечного множества квантовых гармонических осцилляторов в гильбертовом пространстве $F(H)$. Модель, которую мы построим, будет сводится к единичному осциллятору, когда $\operatorname{dim} H=1$. Следующее определение даёт аналог когерентных состояний для $F(H)$.\\

\noindent\textbf{Определение 7.6.} \ Для $f \in H$ элемент $e(f) \in F(H)$, определяемый формулой
$$
e(f)=\sum_{n=0}^{+\infty} \frac{f^{\otimes}}{\sqrt{n !}}
$$называется экспоненииальным вектором.\\

Непосредственно проверяется, что скалярное произведение экспоненциальных векторов равно
$$
\langle e(f), e(g)\rangle_{F(H)}=e^{\langle f, g\rangle_{H}}
$$
\noindent\textbf{Лемма 7.1.} \ \textit{Линейные комбинации экспоненциальных векторов из множсества}
$$
\{e(f), \quad f \in H\}
$$\textit{плотны} в $F(H)$.\\\\
\textbf{Доказательство}. Заметим, что
$$
\left.\frac{d^{n}}{d t^{n}}(e(t f))\right|_{t=0}=\sqrt{n !} f^{\otimes^{n}}
$$
C другой стороны, множество линейных комбинаций элементарных тензоров вида $f^{\otimes^{n}}$ позволяет выразить любой элементарный тензор из $H^{\otimes_{s}^{n}}$. Докажем это по индукции. Для $n=2$ получаем
$$
f \otimes g+g \otimes f=(f+g) \otimes(f+g)-f \otimes f-g \otimes g
$$
Пусть утверждение доказано для $n$. Докажем его для $n+1$. Рассмотрим элемент $h$, представляющий собой симметризованное тензорное произведение $f$ и $g^{\otimes n}$ :
$$
h=f \otimes g^{\otimes^{n}}+g \otimes f \otimes g^{\otimes^{n-1}}+\cdots+g^{\otimes^{n}} \otimes f,
$$

\newpage

\noindent где $f, g \in H$. Утверждение верно, если из равенства нулю скалярного произведения
$$
\left\langle h, u^{\otimes^{n+1}}\right\rangle_{H^{\otimes^{n+1}}}=0
$$
для любого $u \in H$ следует, что
$$
h=0
$$
Заметим, что
$$
\left\langle h, u^{\otimes^{n+1}}\right\rangle_{H^{\otimes^{n+1}}}=(n+1)\langle f, u\rangle_{H}\langle g, u\rangle_{H}^{n}
$$
Положим $u=f$, тогда из (7.4)-(7.5) вытекает $\langle g, f\rangle_{H}=0$ для всех $f \in H$, так что $g=0$. Аналогично, подставляя $u=g$, получаем, что $f=0$. Тем самым $h=0$.

В силу леммы 7.1 любой линейный оператор в $F(H)$ достаточно задать на экспоненциальных векторах. Экспоненциальные вектора обладают ещё одним важным свойством, которое нам потребуется в дальнейшем.\\

\noindent\textbf{Лемма 7.2.} \ \textit{Отображение} $U: F(H \oplus K) \rightarrow F(H) \otimes F(K)$, \textit{Отображение заданное на экспоненциальных векторах формулой}
$$
U(e(f \oplus g))=e(f) \otimes e(g), \ f \in H, \ g \in K,
$$
\textit{является унитарным оператором.}
\end{document}