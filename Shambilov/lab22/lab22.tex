\documentclass[a5paper, 10pt]{book}

\usepackage[left=8mm, top=8mm, right=8mm, bottom=8mm, nohead, nofoot]{geometry}
\usepackage[english, russian]{babel}
\usepackage[T2A]{fontenc}
\usepackage[utf8]{inputenc}
\usepackage{wasysym}
\usepackage{amssymb}

\author{Руслан Шамбилов}
\date{26.02.2023}

\setcounter{page}{98}

\begin{document}
    \begin{center}
            \begin{spacing}
                \textbf{\S\,4. ЧИСЛО E}
                \noindent\rule{\textit}{1pt}
        \end{spacing}
    \end{center}
    \par \textbf{48. Число e как предел последовательности.} Мы используем здесь предельный переход для определения нового, до сих пор не встречавшегося нам числа, которое имеет исключительную важность как для самого анализа, так и для его приложений.
    \par Рассмотрим переменную 
    \[x_n = (1 + \frac{1}{n})^n\]
    и попытаемся применить к ней теорему \textbf{$n^{\circ}$ 44.}
    \par Так как с возрастанием показателя n основание степени здесь убывает, то ''монотонный'' характер переменной непосредственно 

    \newpage

    \begin{center}
            \begin{spacing}
                \small{\S\,4. ЧИСЛО E}
                \noindent\rule{\textit}{1pt}
        \end{spacing}
    \end{center}
    
    \small{не усматривается. Для того чтобы убедиться в нем, прибегнем к разложению по формуле бинома:}
    \begin{eqnarray} 
        x_n = (1 + \frac{1}{n})^n = 1 + n \cdot \frac{1}{n} + \frac{n(n-1)}{1 \cdot 2} \cdot \frac{1}{n^2} + \nonumber\\ +\, \frac{n(n-1)(n-2)}{1 \cdot 2 \cdot 3} \cdot \frac{1}{n^3} + ... + \frac{n(n-1)...(n-k+1)}{1 \cdot 2 \cdot ... \cdot k} \cdot \frac{1}{n^k} + ... \nonumber\\ ... + \frac{n(n-1)...(n-n+1)}{1 \cdot 2 \cdot ... \cdot n} \cdot \frac{1}{n^n} = \nonumber\\ = 1 + 1 + \frac{1}{2!}(1 - \frac{1}{n}) + \frac{1}{3!}(1 - \frac{1}{n})(1 - \frac{2}{n}) + ... \nonumber\\ ... + \frac{1}{k!}(1 - \frac{1}{n})\,...\,(1 - \frac{k-1}{n}) + ... \nonumber\\ ... + \frac{1}{n!}(1 - \frac{1}{n})\,...\,(1 - \frac{n-1}{n}).
    \end{eqnarray}
    \par Если от $x_n$ перейти теперь к $x_{n+1}$, т. е. увеличить n на единицу, то прежде всего добавится новый (n+2)-1 (положительный) член, каждый же из написанных n+1 членов \textit{увеличится}, ибо любой множитель в скобках вида $1 - \frac{s}{n}$ заменится \textit{большим множителем} $1 - \frac{s}{n+1}$. Отсюда и следует, что
    \[x_{n+1} > x_n,\]
    т. е. переменная $x_n$ оказывается \textit{возрастающей}.
    \par Теперь покажем, что она к тому же \textit{ограничена сверху}. Опустив в выражении (1) все множители в скобках, мы этим увеличим его, так что
    \[x_n < 2 + \frac{1}{2!} + \frac{1}{3!} + ... + \frac{1}{n!} = y_n\]
    \par Заменив, далее, каждый множитель в знаменателях дробей (начиная с третьей) числом 2, мы еще увеличим полученное выражение, так что, в свою очередь,
    \[y_n < 2 + \frac{1}{2} + \frac{1}{2^2} + ... + \frac{1}{2^{n-1}}.\]
    Но прогрессия (начинающаяся членом \frac{1}{2}) имеет сумму, меньшую единицы, поэтому $y_n < 3$, а значит и подавно $x_n < 3$.
    \par Отсюда уже следует, по теореме \textbf{$n^{\circ}$ 44.}, что переменная $x_n$ имеет конечный предел. По примеру \textit{Эйлера} его обозначают всегда \textit{буквой e}. \newline Это число 
    \[e = \lim(1 + \frac{1}{n})^n\]

    \newpage

    \begin{center}
            \begin{spacing}
                \small{ГЛ. \romannumeral 3. ТЕОРИЯ ПРЕДЕЛОВ}
                \noindent\rule{\textit}{1pt}
        \end{spacing}
    \end{center}

    мы имели в виду. Вот первые 15 знаков его разложения в десятичную дробь:
    \[e = 2,71828\:18284\:59045\:...\]
    \par Хотя последовательность 
    \begin{eqnarray}
        \small{x_1 = (1 + \frac{1}{1})^1 = 2;\:x_2 = (1 + \frac{1}{2})^2 = 2,25; \nonumber\\ x_3 = (1 + \frac{1}{3})^3 = 2,3703\:...;\,...;\:x_{100} = (1 + \frac{1}{100})^{100} = 2,7048\,...;\,... \nonumber}
    \end{eqnarray}
    \small{и сходится к числу e, но \textit{медленно}, и ею пользоваться для приближенного вычисления числа e - невыгодно. В следующем номере мы изложим удобный прием для этого вычисления, а также попутно докажем, что e есть число иррациональное.}

    \par \textbf{49.\: \small{Приближенное вычисление числа e} Вернемся к равенству (1). Если \textit{фиксировать k} и, считая n > k, отбросить все члены последней части, следующие за (k+1)-м, то получим неравенство}
    \begin{eqnarray}
        \small{x_n > 2 + \frac{1}{2!}(1 - \frac{1}{n}) + \frac{1}{3!}(1 - \frac{1}{n})(1 - \frac{2}{n}) + ... \nonumber\\ ... + \frac{1}{k!}(1 - \frac{1}{n})\:...\:(1 - \frac{k-1}{n}). \nonumber}
    \end{eqnarray}

    Увеличивая здесь n до бесконечности, перейдем к пределу; так как все скобки имеют пределом единицу, то найдем:
    \[e \geq 2 + \frac{1}{2!} + \frac{1}{3!} + ... + \frac{1}{k!} = y_k.\]
    Это неравенство имеет место при любом натуральном k. Таким образом, имеем
    \[x_n < y_n \leq e,\]
    Откуда ясно [в силу теоремы 3) \textbf{$n^{\circ}$ 38}], что и 
    \[\lim y_n = e.\]
    \par \small{Переменная $y_n$ для приближенного вычисления числа e гораздо удобнее, чем $x_n$. Оценим степень близости $y_n$ к e. С этой целью рассмотрим сначала разность между любым значением $y_{n+m}(m = 1, 2, 3,\:...)$ слеующим за $y_n$, и самим $y_n$. Имеем}
    \begin{eqnarray}
        y_{n+m} - y_n = \frac{1}{(n+1)!} + \frac{1}{(n+2)!} + ... + \frac{1}{(n+m)!} = \nonumber\\ = \frac{1}{(n+1)!}(1 + \frac{1}{n+2} + \frac{1}{(n+2)(n+3)} + ... + \frac{1}{(n+2)(n+3)\:...\:(n+m))}. \nonumber
    \end{eqnarray}
    Если в скобках (...) заменить все множители в знаменателях дробей через n+2, то получим \textit{неравенство}
    \small{\[y_{n+m} - y_n < \frac{1}{(n+1)!}(1 + \frac{1}{n+2} + \frac{1}{(n+2)^2} +\:...\: + \frac{1}{(n+2)^{m-1}}),\]}

    \newpage

    \begin{center}
            \begin{spacing}
                \small{\S\,4. ЧИСЛО E}
                \noindent\rule{\textit}{1pt}
        \end{spacing}
    \end{center}

    которое лишь усилится, если заменить скобки суммой бесконечной прогрессии:
    \[y_{n+m} - y_n < \frac{1}{(n+1)!} \cdot \frac{n+2}{n+1}.\]
    \par Сохраняя здесь n неизменным, станем увеличивать m до бесконечности; переменная $y_{n+m}$ (занумерованная значком m) принимает последовательность значений
    \[y_{n+1},\:y_{n+2},\:...\:,y_{n+m},\:...,\]
    очевидно сходящуюся к e. Поэтому получаем в пределе
    \[e - y_n \geq \frac{1}{(n+1)!} \cdot \frac{n+2}{n+1}\]
    или, наконец,
    \[0 < e - y_n < \frac{1}{n!n}*).\]
    \par Если через o обозначить отношение разности $e - y_n$ к числу $\frac{1}{n!n}$ (оно, очевидно, содержится между нулем и единицей), то можно написать также
    \[e - y_n = \frac{o}{n!n}.\]
    Заменяя здесь $y_n$ его развернутым выражением, мы и придем к важной формуле:
    \begin{eqnarray}
        e = 1 + \frac{1}{1!} + \frac{1}{2!} + \frac{1}{3!} + ... + \frac{1}{n!} + \frac{o}{n!n},
    \end{eqnarray}
    которая послужит отправной точкой для вычисления e. Отбрасывая последний, ''дополнительный'', член и заменяя каждый из оставленных членов его десятичным приближением, мы и получим приближенное значение для e
\end{document}
